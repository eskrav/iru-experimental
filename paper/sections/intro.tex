Pragmatic theories and theories of language processing typically include
constraints against elements which add no new information to the
discourse, or are linguistically or informationally redundant (cf.
\citealp{Aylett2004}, for a theory of phonetic reduction; \citealp{Cohen1978},
for a computational theory of speech act generation; \citealp{Grice1975}, for a
theory of rational communication; or \citealp{Jaeger2010}, for a theory of
reduction at all levels of linguistic representation, among many
others). At the form level, redundancy may include overt mention of, or
increased articulatory effort towards producing material that is easily
predictable or recoverable in context. In other words, more signal is
provided than the comprehender requires to accurately recover the
intended phonological, lexical, or syntactic form. Examples of
redundancy avoidance at this level include vowel shortening \citep{Aylett2004}, use of shorter word variants \citep{Mahowald2013}, or omission of optional complementizers
\citep{Jaeger2010}.

At the informational or conceptual level, redundancy refers to the
explicit mention of information that the comprehender is already in a
position to infer automatically, using world knowledge or common ground
information, or that is already entailed or strongly implied by the
preceding discourse. In other words, more information is provided than
needed to recover the intended meaning or world state. In contrast to
redundancy avoidance at the form level, constraints against
\emph{overinformativeness}, or redundancy, at the informational level
have always been somewhat debated \citep{Grice1975}.

There is ample evidence that speakers are routinely overinformative at
the informational level, and that speaker overinformativity is
frequently tolerated by listeners \citep{Baker2008,
Engelhardt2006, Nadig2002, Walker1993}. In this paper, we explore the question of whether there is
empirical evidence for constraints against this variety of speaker
redundancy, when they might come into play, and how comprehenders may
react to and interpret any violations of such constraints (or at least,
of their expectations that speakers will be concise). Specifically, we
look at cases where the redundancy is at the level of background world
knowledge -- as opposed to, for example, repeating something that has
already been stated, or referring to an object in more descriptive
detail than strictly necessary given the physical context.

Utterances such as (1) are at face value redundant, in that they overtly
state that \enquote{John} \emph{paid the cashier}, which conventionally
can be inferred simply on the basis of him having gone \emph{shopping}.

\ex. John went grocery shopping. He paid the cashier!

Once it has been established that \emph{John} went grocery shopping,
comprehenders' expectations of a world state where the \emph{paying}
action has occurred are very high (\citealp{Bower1979}; cf.
\citealp{Zwaan1995}). A theoretic account of utterance
choice which places a constraint on informational redundancy would
predict that uttering the second sentence in this context would be
marked, at best. Further, it would predict that comprehenders should
note this markedness, and possibly penalize it, or at least regard the
speaker as somewhat odd. However, in this paper we also show that
comprehenders, through pragmatic reasoning about the common ground, can
accommodate these utterances by changing their previous beliefs about
the likely world state.

\subsection{World knowledge}\label{world-knowledge}

Utterances like the one shown in (1) are redundant on the basis of
background world knowledge. As background knowledge is fairly
unsystematic and comprehender-specific, and can be difficult to control
for, here we use \emph{script}, or \emph{schema} knowledge as a proxy
for world knowledge. \emph{Script} knowledge refers to people's implicit
awareness of the typical event structures of various stereotyped
activities, such as \emph{going shopping} or \emph{going to a
restaurant} \citep{Fillmore2006, Minsky1975, Schank1977}. The
former, for example, normally involves events such as \emph{going to a
store}, \emph{selecting food items}, and \emph{paying the cashier}.
Comprehenders anticipate upcoming events once a script is
\enquote{invoked} \citep{Zwaan1995}; and when recalling stories based
on scripts, have difficulty distinguishing actions that were actually
mentioned, and those that were unmentioned but implied by the script
\citep{Bower1979}. These findings suggest that events which are
strongly associated with a script are almost part of its conventional
meaning, and that explicitly mentioning their occurrence is therefore
redundant\footnote{Highly inferable events are occasionally used as
  temporal anchors (\emph{After she entered the restaurant,
  she}\ldots{}), and may be used to transition back from interruptions
  to the script (\emph{She stopped to talk to Brad on the street. She
  then entered the restaurant}\ldots{}). However, outside of these
  contexts, easily inferable script events are usually not mentioned
  overtly \citep{Bower1979, Regneri2010}.}.

Utterance (1) introduces a well-known script or event sequence
(\emph{grocery shopping}), followed by an informationally redundant
event description (\emph{he paid the cashier!}), which references a
highly predictable sub-event from the script. In this example, the event
described in the second sentence is already strongly implied to have
occurred by the preceding invocation of the \emph{grocery shopping}
script -- given the assumption, shared by most speakers and
comprehenders, that people overwhelmingly pay cashiers when they go
grocery shopping. Mentioning it explicitly, therefore, is redundant.

\subsection{Informational redundancy}\label{informational-redundancy}

While most pragmatic theories do address cases where a speaker may be
informationally redundant \citep[][,
among many others]{Grice1975, Horn1984, Levinson2000}, they often leave open the question of whether
comprehenders do, in fact, perceive (unjustified) redundancy as
infelicitous, as well as how they interpret redundant utterances. Most
accounts do argue that comprehenders expect speakers to behave
rationally -- namely, by communicating in a way that is consistent with
getting across the intended message (which, furthermore, should be
truthful). However, as \citet{Grice1975} notes, it's unclear whether
excessive redundancy comes into any real conflict with the goal of
successful (truthful, sufficiently informative, relevant, etc.)
communication -- although comprehenders may wonder what the
\enquote{point} of excessive information is, and attempt to rationalize
unexpected \enquote{dips} in informational utility by infusing them with
additional pragmatic meaning. Informationally redundant utterances do
not clearly interfere with comprehension, as \emph{underinformativeness}
or underspecification does, and may aid comprehension in some cases
\citep[e.g., object identification; cf.][]{Nadig2002,
RubioFernandez2016}\footnote{This is not to say that comprehension is
  not in any way impaired by redundancy, and in fact we suspect that it
  is - but at face value, there is nothing about receiving more
  information than needed that necessarily hinders one from arriving at
  the intended meaning of a message.}. In this light, it is not
straightforwardly clear whether overinformativeness constitutes
non-rational speaker behavior, and specifically to what degree this part
of the \emph{Quantity} maxim holds: \enquote{do not make your
contribution more informative than is required.}\footnote{\citet{Grice1975}
  explicitly suspected that it did not; later accounts separated the
  notions of semantic vs.~form-based informativeness, which is discussed
  in the following section.}

It is, however, possible that comprehenders perceive excessive
information as, at minimum, non-relevant to the discourse \citep{Grice1975,
Horn1984}. The question, then, is whether comprehenders make any
particular note of redundancy, simply find it odd or infelicitous, or
attempt to accommodate it. If comprehenders do perceive redundant
information as irrelevant, then rational speakers should avoid overtly
stating conceptually redundant information, except in those cases where
this information is intended to communicate a more informative
non-literal meaning (or signal an unusual world state). Correspondingly,
comprehenders where possible ought to interpret conceptually redundant
utterances as either an attempt to convey some non-literal (relevant and
informative) meaning, or as reflecting a background world state where
the information conveyed can't be taken for granted, and is therefore
informative. How comprehenders do in fact react to redundancy has to
date only been empirically investigated within the relatively narrow
scope of nominal modification, where the evidence, discussed further in
Section \ref{related-work}, has largely been equivocal.

Ultimately, the question of how comprehenders treat overinformativeness
is relevant to a more general theory of human communication, and should
be answered to determine the extent to which: a) comprehenders and/or
speakers consistently behave in a \enquote{Gricean} manner; b) under
which conditions they do so, and which deviations from communicative
norms are more likely to occur/be tolerated; and c) to what extent
comprehenders attempt to resolve apparent violations, and which
strategies they use to do so. If comprehenders do not appear to make
much of overinformativeness (whether in terms of inferences made, or
maxim violations perceived), and there is little evidence that speakers
deliberately use overinformative utterances to convey specific
non-literal meanings, then it's questionable to what degree
overinformativeness violates communicative norms, in the first place.

\subsection{How might comprehenders react to informational redundancy? }\label{how-might-comprehenders-react-to-informational-redundancy}

In this section, we speculate in more detail how comprehenders might
react to specific instances of informational redundancy, or
\emph{overinformativeness}. We distinguish three theoretical
possibilities. Specifically, we will consider what might happen when a
comprehender encounters one of our experimental utterances (normally
embedded within a larger context):

\ex. John just came back from the grocery store. \textbf{He paid the
  cashier.}

\subsubsection{Hypothesis 1: No
inference}\label{hypothesis-1-no-inference}

The first possibility is that comprehenders do not find informational
redundancy particularly marked, as it does not necessarily interfere
with interpreting the intended message -- or, at most, find redundant
utterances slightly odd or suboptimal, as has been found in some studies
\citep{Davies2010}. It's both likely that comprehenders do not
expect speakers to exhibit entirely rational communicative behavior at
all times, and that conversational maxims, if they have little or no
effect on the comprehender's ability to understand the basic meaning of
an utterance, are followed only insofar as the speaker feels like it.
Similarly, comprehenders may note redundancy in speech, but not draw any
inferences regarding some intended meaning or background world state.
They might instead ascribe the redundancy to some kind of speaker error:
perhaps the speaker is stalling for something else to say, having
production difficulty, or is simply not communicating very well in that
particular instance. In the case of our utterance (2), in this scenario,
we might expect that comprehenders would interpret the utterance
literally, and make no more of it than stated; i.e., they would take
away the message that on some particular instance, John paid the
cashier, and perhaps the speaker described it in a bit more detail than
strictly necessary.

\subsubsection{Hypothesis 2:
Non-detachability}\label{hypothesis-2-non-detachability}

If comprehenders do expect speaker utterances to always have a certain
level of informational utility, then they may attempt to resolve the
provision of excessive or unnecessary information by drawing pragmatic
inferences, regarding what they think the utterance means or signifies
from the speaker's perspective. These pragmatic inferences would then
serve to increase the informational utility of the utterance, and allow
comprehenders to maintain the belief that the speaker is being
cooperative -- since assigning an \enquote{informative} pragmatic
meaning to an apparently redundant utterance in effect removes the
redundancy. In the case of utterance (2), comprehenders might conclude
that John's \emph{cashier-paying} is being announced due to its being
unusual or unexpected, and that John can't therefore typically be
counted on to pay the cashier. This reaction should occur as long as the
background and linguistic context is basically consistent with that
interpretation, and, as in the case of most pragmatic inferences, should
be unaffected by changes to the utterance which do not alter its
semantic content \citep[generally referred to as \emph{non-detachability};][]{Grice1975}, such as prosodic and/or discourse markers which do not
change the truth conditions of the sentence - i.e., the inference should
be attached to the semantic content, not the specific linguistic form of
the utterance.

\subsubsection{Hypothesis 3: Form
sensitivity}\label{hypothesis-3-form-sensitivity}

The third possibility is that, as in Hypothesis 2, comprehenders react
to a statement of \emph{John's} having paid the cashier by inferring
that \emph{John} must be a habitual non-payer. However, as the
inferences we are concerned with are based, in a sense, on the specific
\emph{form} of the utterance (i.e., too much signal is used to
communicate something that would have already been understood), it is
possible that such inferences may be relatively sensitive to how exactly
the utterance is expressed\footnote{The principle of
  \emph{non-detachability} generally does not hold for Manner
  implicatures \citep{Grice1975}, which are conceptually similar to the
  inferences we look at here; for space reasons, we will not go into
  more detail on this type of implicature, however.}. In particular, we
suspect that expending extra articulatory effort on expressing our
already redundant utterance would increase the strength of any pragmatic
inferences drawn (or even cause inferences to be drawn where none would
be otherwise). Fundamentally, a greater show of \emph{intentionality},
in apparently violating a maxim, provides more evidence that the maxim
is not simply being violated due to a speech error or difficulty in
utterance planning, and signals to the comprehender that more should be
made out of the apparent violation. This also echoes \citet{Wilson2004}'s stated basis for their Communicative Principle of Relevance:
\enquote{by producing an ostensive stimulus, the communicator therefore
encourages her audience to presume that it is relevant enough to be
worth processing} (an \emph{ostensive stimulus} being one that is
\enquote{designed to attract an audience's attention and focus it on the
communicator's meaning}). In the case of our utterance, what we would
predict in this case is that the more obvious effort is expended on
producing the utterance (whether in the form of prosodic stress, or
another attention-drawing signal of relevance and intentionality), the
stronger the inference. To note, some amount of form sensitivity is not
necessarily incompatible with the second hypothesis
(\emph{non-detachability}), but the complete absence of an inference
would be.

In this paper, we present three experiments, run concurrently on the
same population, which test whether informationally redundant event
descriptions lead comprehenders to alter initial beliefs about how
predictable, or habitual, the event in question is, on the premise that
less habitual events are more likely to be mentioned. We predict,
consistent with the first and third scenario outlined above, that
informationally redundant event descriptions should generate what we
term \emph{non-habituality inferences}, where comprehenders resolve the
apparent dip in informational utility by concluding that the usually
predictable and habitual event described is in fact relatively
\emph{non-habitual}/non-predictable, as this would justify its mention.
For example, in our scenario involving \emph{shopping} and \emph{paying
the cashier}, a possible inference would be that \enquote{\emph{John}}
does not pay habitually (e.g., has someone else pay for him, is a
habitual shoplifter, or gets free groceries). We first look at these
utterances in discourse contexts which implicitly support a
\emph{non-habituality} inference, through semantically vacuous prosodic
or discourse markers which serve to draw the listener's attention to the
information being conveyed\footnote{When communicating, speakers
  frequently use multiple cues to make an inference maximally easy to
  compute for the comprehender (consider, for example, the difficulty of
  interpreting sarcasm without supportive prosody). We therefore
  consider the presence of some amount of prosodic or discourse support
  to be the \enquote{default} case.}. The first experiment uses implicit
exclamatory prosody (the marker \enquote{\emph{!}}) to signal that the
utterance is an intentionally conveyed, important, and relevant piece of
information. The second experiment uses the discourse marker
\enquote{\emph{oh yeah, and\ldots{}}} to do the same, while avoiding the
surprise conventionally implied by the exclamation mark. In the third
experiment, we predict that informational redundancy by itself, in
absence of prosodic or discourse cues as to relevance and
intentionality, triggers weaker \emph{non-habituality} inferences,
consistent with the third hypothesis (\emph{form sensitivity}).

In the next section we review relevant literature, and in the following
sections we describe the three experiments we ran.