Our work builds on two primary strains of research: interpretation and
perception of informational redundancy on the one hand, and relatively
new work on inferences about background world states (vs.~speaker
intentions) on the other. We also look at the effects of implicit
prosody on pragmatic interpretation, which to date has largely been
limited to semantic effects of contrastive prosody. Further, we overall
look at the (systematic) interpretation of \emph{particularized}, or
\emph{ad-hoc} pragmatic inferences, which arise only in specific
contexts, and/or on the basis of reasoning about world knowledge. These
have not received a lot of attention from pragmaticists, experimentally
or otherwise, partially due to their idiosyncratic nature, making them
difficult to study systematically; and partially due to being seen as
less relevant to a theory of pragmatic vs.~semantic meaning than, for
example, scalar implicatures \citep{Levinson2000}.

\subsection{\texorpdfstring{The problem of
\emph{overinformativeness}}{The problem of overinformativeness}}\label{the-problem-of-overinformativeness}

First, we want to discuss a problem of terminology. In most experimental
work, informational redundancy has been described as a problem of
\emph{overinformativeness}, \emph{overspecification} or
\emph{overdescription}, and as addressed by the second part of Grice's
Quantity Maxim, which states that speakers should provide no more
information than is necessary to get their message across. However,
\emph{overinformativeness} in the pragmatic literature has been used to
refer to both informational redundancy \citep{Engelhardt2006, Grice1975}, as well as to the relative informativeness of terms in an
implicational scale (e.g., the use of \emph{some} when \emph{all} is
sufficient) \citep{Horn1984, Levinson2000}. The latter variety of
\emph{overinformativeness}, now more typically associated with the
Quantity Maxim, is more a problem of unjustified vagueness where a more
\emph{precise} description is available. Informational redundancy, in
contrast, is a problem of \emph{excessive} wordiness or precision, as in
the case of overinformative nominal modification (such as using
\emph{the big red cup} or \emph{the cup on the towel} to identify the
only available cup in a given context), where speakers might choose to
describe objects in more detail than is strictly necessary. In this
paper we concern ourselves strictly with overinformativeness in the
sense of informational redundancy, as originally described by \citet{Grice1975}, and in the literature on nominal overspecification.

While informationally redundant utterances are typically regarded as
infelicitous in the linguistics literature, they have been observed to
be surprisingly common in natural dialog. \citet{Baker2008} observed
that such utterances are frequently used in response to signs of
listener non-comprehension, when responding to listener questions, or
when speaking to strangers. \citet{Walker1993} also concludes that
informationally redundant utterances are specifically used to address
cognitive resource limitations (e.g., memory for preceding discourse,
limited inference-making capacity), as well as to serve a narrative
function. In the latter case, this may for example involve drawing
attention to a particularly salient or relevant fact. In other words,
many or most informationally redundant utterances are not in fact
redundant, as the apparent redundancy has communicative purpose.
Literature on nominal overspecification has similarly found that
speakers are extremely likely to attach \enquote{redundant} color
descriptions to nouns, even when doing so provides no new information
regarding which object is being referred to. However, in this case as
well, there is evidence that most \emph{overinformative} nominal
modification is not in fact \emph{overinformative}, as
\enquote{overdescribing} an object can facilitate more rapid and
efficient object identification. Here we will review some of the
experimental work on informational redundancy, with a focus on
interpretation of nominal overspecification.

Most experimental work on production and comprehension of
informationally redundant utterances has focused on nominal modification
in referent identification tasks, which typically instruct participants
to look at or somehow engage with items such as: \emph{the {[}red{]}
apple}, \emph{the {[}tall{]} boot} \citep{Davies2010, Davies2013, Engelhardt2006, Nadig2002, Pogue2016, Sedivy2003}. The aim of these studies has been to
determine some combination of the following: 1) whether overinformative
descriptions are perceived as infelicitous by comprehenders (i.e.,
whether overinformativeness apparently violates some communicative
norm); 2) whether overinformativeness helps, hinders, or has no effect
on object identification; 3) whether comprehenders attempt to
accommodate overinformative descriptions by making inferences which
increase the informational utility of the descriptions; and 4) whether
comprehenders alter their beliefs about the rationality of the speaker
(or the baseline informativeness of their speech) following use of
overinformative descriptions.

What has been found is that in interactive, spontaneous speech, speakers
frequently modify nouns with adjectives that are not strictly necessary
for referent identification (e.g., referring to a cup as \emph{the red
cup}, in a context where there are no other cups of any color)
\citep[][: 30\% and 50\% of nominal descriptions were overspecified in spontaneous speech,
respectively]{Engelhardt2006, Nadig2002}. Further, comprehenders frequently do not find such
utterances infelicitous: \citet{Engelhardt2006} showed that
comprehenders judge overinformative descriptions as significantly more
acceptable than underinformative descriptions, and that overinformative
descriptions do not trigger additional (e.g., contrastive) inferences.
\citet{Davies2010} find that overinformative expressions are more
likely to be produced, and less likely to be judged infelicitous, than
underinformative expressions, although they are still judged to be
suboptimal\footnote{However, Davies and Katsos purposefully use
  adjectives less likely to be produced spontaneously - color
  adjectives, by far the most likely to be used redundantly, are
  avoided, and the adjectives that they use are largely either
  inherently contrastive (e.g., \enquote{tall,} \enquote{big}); or
  describe a default, assumed state (e.g., \enquote{unbroken egg,}
  \enquote{fresh apple}).}. \citet{Sedivy2003} showed that when comprehenders
hear an object described with a clearly overinformative and prototypical
color adjective (e.g., \enquote{yellow banana}), they make contrastive
inferences (e.g., rapidly infer that a non-yellow banana must also be
present).

What seems to emerge is that overinformative descriptions are easily
tolerated when they describe perceptually useful or non-canonical
properties, which may speed up object identification; and are more
likely to be judged suboptimal, and/or trigger pragmatic inferences,
when they don't. Indeed, \citet{RubioFernandez2016} showed that
experimentally increasing the perceptual usefulness of color adjectives
causes them to be produced more frequently, as well as that color
adjectives are more likely to be used for atypical than typical colors.
In a related line of research, \citet{Pogue2016} found that after
being exposed to a speaker repeatedly using overinformative (color or
scalar) object descriptions, comprehenders are less likely to make
generalization about the speaker's rationality or informativity than
when they use underinformative descriptions. This suggests that
comprehenders are relatively insensitive to deviations from
\enquote{optimal} informativity that do not interfere with basic
utterance comprehension, or else perceive them as relatively commonplace
and inconsequential.

Overall, this work has shown that some types of informational redundancy
may be helpful to the comprehender, and that informational redundancy in
general is tolerated by comprehenders. There is, however, also evidence
that informationally redundant utterances which have no apparent (e.g.,
perceptual) utility are unlikely to be produced, are generally judged to
be relatively infelicitous, and tend to generate inferences. More
generally, there is still some difficulty in distinguishing what
constitutes informational redundancy, which creates difficulty in
determining the precise theoretical implications of previous work (e.g.,
perceptually helpful \enquote{redundant} adjectives are questionably
redundant in the first place, in the sense of having communicative
utility). Additionally, these studies are limited by the fact that they
uniformly focus on a very particular, and relatively concise variety of
informational redundancy, which is further bound to a specific class of
lexical items, raising the question of to what degree it's possible to
generalize from the results. What this points towards is a need to look
at informational redundancy in the context of utterances and
constructions that are both quite costly for speakers, and have no
readily apparent utility to comprehenders - either in terms of
perception or comprehension, or in terms of facilitating the completion
of a task. Further, we would argue that it's important to investigate
constructions that are less bound to a specific set of lexical items,
and are more likely to be perceived as flouting of a conversational norm
against redundancy - for example, complex and lengthy multi-word
utterances such as those in Example (2).

\subsection{Common ground assumptions}\label{common-ground-assumptions}

To date there has been relatively little work on the different
strategies comprehenders might employ in making sense of an apparent
violations of conversational maxims. Most work has focused on the
scenario where a comprehender detects an apparent maxim violation,
assumes that the speaker is in fact being cooperative, and comes up with
an additional, non-literal meaning that the speaker may have intended
(which repairs the apparent violation). Another strategy is simply to
assume that the speaker is being plainly uncooperative, or that there is
an intended meaning but that the comprehender is not privy to it, if no
plausible intended meaning can be computed. A third strategy, which has
received little attention, is that of modifying background assumptions
about the world in which events take place, if doing so would repair the
apparent violation. The lack of attention to this strategy is likely
partially due to a focus on implicatures, or specifically intended
meanings. To our knowledge, the only work to look at this in depth is
\citet{Degen2015a}, which investigated comprehenders'
willingness to revise their assumptions about the assumed common ground,
in response to utterances whose pragmatic meaning would otherwise be
inconsistent with it. They found that background assumptions about the
world are surprisingly defeasible: comprehenders frequently accommodate
the pragmatic meaning of utterances such as \enquote{\emph{\textbf{some}
of the marbles sank}} (upon being thrown into a pool), by assuming that
the utterances signify a strange scenario where physics doesn't quite
work as expected. Further, a pre-utterance belief that a scenario is
strange significantly increases the strength of the \enquote{\emph{some,
but not not all}} implicature that is then drawn by the comprehender.

In our case, if, as in Example (1), a speaker states that \emph{John},
having gone shopping, \emph{paid the cashier}, a comprehender might
\enquote{repair} the redundancy by assuming that \emph{John} does not in
fact habitually pay the cashier. While this may occur parallel to an
assumption that the speaker \emph{intended} to use this utterance to
communicate that \emph{John} is not a cashier-paying individual, the
strategy of modifying background assumptions can well proceed without
any assumptions about speaker intent. Perhaps the comprehender is a
third party not privy to the background knowledge of the speaker and
intended listener, or perhaps the speaker isn't aware that the listener
isn't familiar with \emph{John}'s usual paying habits. In fact, in the
case of our example, it seems relatively unlikely that a speaker would
choose to communicate information about \emph{John's} paying habits in
this particular manner, making this an inference, but not an
implicature. While most theoretical interest lies in implicatures, it's
important to be able to model pre-utterance and changing post-utterance
assumptions about the common ground, given that they have been
demonstrated to have a marked effect on which inferences are drawn by
comprehenders, as well as their strength (\citealp{Degen2015a}; see also
the literature on presupposition, e.g.: \citealp{Stalnaker1973}). Further, an
exclusive focus on intended meanings, rather than changes in background
assumptions, may lead to erroneous conclusions that comprehenders are
drawing no pragmatic inferences from an given utterance. In our paper,
we explore a method for testing the shifting of background assumptions,
collect data that can be used in the future to test formal models of
pragmatic reasoning, and explore the willingness of comprehenders to
shift background assumptions in different contexts.

\subsection{Effect of implicit prosody on pragmatic
interpretation}\label{effect-of-implicit-prosody-on-pragmatic-interpretation}

One of the questions we ask, relevant both to accurately detecting and
modeling an effect of informational redundancy, is to what degree
increased emphasis on the utterance (without changing the semantic
content, or propositional value) influences the interpretation of
informational redundancy. Although it is generally accepted that
prosodic emphasis may influence utterance interpretation, there is very
little empirical evidence that prosodic changes which contribute little
by way of conventional meaning have a substantial effect on generation
of pragmatic inferences\footnote{An exception is the effect of
  contrastive prosody \citep[e.g.,][]{Kurumada2012}, which
  is generally thought to be semantic -- however, it has also been
  suggested that the effect of contrastive prosody is a pragmatic
  inference, as discussed in the following paragraph.}. One can however
imagine that a redundant statement made loudly and confidently may lead
a comprehender to believe that the speaker is very intentionally
communicating that particular bit of information to them \citep[cf.]{Wilson2004}, and that it should be taken seriously (signifying either
that the speaker is being blatantly uncooperative by violating a
communicative norm for no reason, or that there is an additional reason
that the information was so purposefully transmitted). On the other
hand, if a speaker vaguely mumbles an informationally redundant
utterance under their breath, the comprehender might simply conclude
that the speaker is reminding themselves of something, is unsure about
what they really want to say, is mentally rehearsing a course of events,
having some production difficulty, etc..

Along these lines, \citet{Bergen2015} hypothesize, on the basis
of formal probabilistic models of pragmatic reasoning, that rather than
focal/contrastive stress carrying conventional semantic meaning, the
contrastive or exhaustive interpretation (\enquote{\emph{\textbf{BOB}
went to the movies}} -\textgreater{} \textbf{only} \emph{Bob} went to
the movies) arises due to the comprehender perceiving the speaker as
having made extra effort to communicate exactly that particular bit of
info to them. They argue that an utterance which is increased in volume
or duration is more likely to be attended to or accurately perceived by
the comprehender and that, correpondingly, speakers can intentionally
exploit this to signal to comprehenders that this particular utterance
is important, and specifically meant to not be confused with any
alternative utterances. On the basis of this and similar work, we
therefore experiment with having participants interpret an
informationally redundant utterance both with implicit exclamatory
prosody (ending with an exclamation mark), as well as with no implicit
prosody (ending simply with a full stop).

\subsection{Context-dependent
implicatures}\label{context-dependent-implicatures}

To date, most formal or experimental research on pragmatic inferences
has focused on the production and interpretation of scalar implicatures
\citep{Horn1984,Levinson2000}, such as the use of \emph{some} to
implicate \emph{not all}, or \emph{warm} to implicate \emph{not hot}.
Non-generalized \emph{ad-hoc} inferences, which arise only in specific
contexts, have not received much attention from pragmaticists,
experimentally or otherwise. Traditionally, scalar implicatures have
been regarded as a separate class of \emph{conventionalized} inferences
which rely minimally on context or general reasoning about speaker
intentions \citep{Levinson2000}, and which arise from the use of specific
lexical items (or classes of lexical items). In recent years this view
has increasingly been challenged \citep{Degen2016, Grodner2010}, with evidence indicating
that the distinction between \emph{conventionalized}
(\emph{generalized}) inferences, and \emph{particularized}
(\emph{ad-hoc}) inferences is in any case not categorical, although the
nature of differences between the two classes remains difficult to
determine, and the latter case has traditionally been understudied.

Research on conventionalized inferences has been critical to developing
formal linguistic theory, due to the role they play in disambiguating
pragmatic and semantic contributions to utterance meaning. However,
context-dependent (\emph{ad-hoc}) inferences, which occur far more
frequently and ubiquitously, are similarly important to developing a
more general theory of human communication \citep[as originally intended by][]{Grice1975}. The body of experimental work teasing apart which
properties of utterances trigger, alter, or modulate the strength of
pragmatic inferences is still relatively small -- however, having a more
comprehensive model of cues which are taken into account by
comprehenders, when interpreting utterances, is necessary both for
building models of pragmatic reasoning, and for interpreting empirical
results. In addition, there is a general need for further quantitative
data on the specific conditions under which inferences are generated, in
order to develop and test predictions of formal models of pragmatic
reasoning \citep[cf.][]{Frank2012}.

In sum, we argue that further empirical work on \emph{ad-hoc} inferences
and informational redundancy on the one hand, and contextual cues which
modulate inferences drawn by comprehenders on the other, is important
for developing comprehensive and predictive theories of communication,
and for developing formal quantitative models of pragmatic reasoning.